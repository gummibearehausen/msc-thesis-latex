\mbox{}
\clearpage
\setstretch{1.3}  % Reset the line-spacing to 1.3 for body text (if it has changed)

% The Abstract Page
\addtotoc{Abstract}  % Add the "Abstract" page entry to the Contents
\abstract{
\addtocontents{toc}{\vspace{1em}}  % Add a gap in the Contents, for aesthetics
Inductive Logic Programming (ILP) is a consolidated technique for mining first-order rules in knowledge
bases.
Nevertheless, it's inherently expensive and the hypothesis search space grows combinatorially with the knowledge base
size. 
This problem is even more dramatic when literals with constants are considered. Therefore, it usually requires
very aggressive pruning to make it feasible. 
Moreover, searching for rules with numerical intervals requires very expensive queries, and frequently does not bring
any significant confidence gain.
We propose a preprocessing step and an extension to ILP top-down algorithm, with the objective of efficiently learning
rules with interesting intervals for numerical attributes, i.e., rules which present a significant gain when restricting
a numerical attribute variable to a specific interval.
In the preprocessing step, we build a lattice that expresses the correlations between a root numerical attribute and
multiple categorical relations and their constants, while in the refinement step of ILP, we query this lattice in order
to predict whether a rule has any interesting numerical interval, as well as obtain a list of refinement
suggestions ordered by an interestingness measure.
This thesis discusses how to efficiently build the correlation lattice, how to incorporate it in the core ILP learning
algorithm, and compares different iterestingness measures. Also, the approach we proposed was evaluated experimentally
on large-scale Linked Data sources.


}

\clearpage  % Abstract ended, start a new page