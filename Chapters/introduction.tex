\chapter{Introduction}
\label{ch:intro}

In the last years, the volume of semantic data available, in particular RDF, has dramatically increased. Initiatives like the W3C Semantic Web, which provides a common standard that allows data to be shared and reused across different applications, and the Linked Open Data, which provides linkages between different datasets that were not originally interconnected, have great contribution in such development. Moreover, advances in information extraction have also made strong contribution, by crawling multiple non-structured resources in the Web and extracting RDF facts.

Nevertheless, information extraction still has its limitations and many of sources might contain contradictory or uncertain information. Therefore, many of the extracted datasets suffer from incompleteness, noise and uncertainty.

In order to reduce such problems, one can apply to the knowledge base a set of inference rules that describes its domain. With that, it's possible to resolve contradictions or strengthen or weaken their confidence values. It's also possible to derive new facts that are originally not existent due to incompleteness. Such inference rules can be of two types:

\begin{enumerate}

 \item \emph{Hard Rules}: Consistency constraints which represent which might represent functional dependencies, functional or inverse-functional properties of predicates or Mutual exclusion. For example:
    \begin{itemize}
      \item \begin{math} marriedTo(x,y) \leftarrow marriedTo(y,x),(x \neq y)\end{math}
      \item \begin{math} grandChildOf(x,y) \leftarrow childOf(x,z),childOf(z,y)\end{math}
      \item \begin{math} parentOf(x,y) \leftarrow childOf(y,x)\end{math}
      \item \begin{math} (z=y) \leftarrow wasBornIn(x,z),wasBornIn(x,y)\end{math}
    \end{itemize}

 \item \emph{Soft Rules}: Weighted rules that frequently, but not always hold in the real world. As they might also produce incorrect information, each rule itself must have a confidence value which should be applied to derived facts, for example married people live in the same place as their partner has confidence 0.8:
    \begin{center}
      \begin{math} livesIn(x,y) \leftarrow marriedTo(x,z)livesIn(z,y)\end{math} [0.8]
    \end{center}
\end{enumerate}

So, if we have an incomplete knowledge base, which lacks information about where \emph{Michelle Obama}l lives, but we know that she's married to \emph{Barack Obama} and he lives in \emph{Washington, D.C.}, both with confidence 1, we could then apply this soft rule to derive the fact \emph{livesIn(MichelleObama, WashingtonDC)} with confidence 0.9.

Such rules are rarely known beforehand, or are too expensive to be manually extracted. Nevertheless, the data itself can be used to mine these rules using \emph{Inductive Logic Programming (ILP)}. 

%http://people.csail.mit.edu/kersting/profile/PROFILE_ilp.html

ILP is a well-established framework for inductively learning relational descriptions (in the form of logic programs) from examples and background knowledge. Given a logical database of facts, an ILP system will generate hypothesis in a pre-determined order and test them against the examples. However in a large knowledge base, ILP becomes too expensive as the search space grows combinatorially with the knowledge base size and the larger the number of examples, the more expensive it is to test each of the hypothesis.

rules with constants might be really interesting.


Moreover, testing hypothesis with constants increases the search space dramatically, making it unfeasible to test all possible hypothesis with constants in a large knowledge base. In such case, it's necessary to arbitrarily reduce the search by restricting the set of constants to be included in ...

data mining, rule mining, datalog rules



talk a bit about ilp

\section{Motivation}
Given the huge size of search space and the great interestingness of rules with constants, we need to smartly prune constants or combinations of constants that  , learning datalog rules can be already extremely costly.

Numerical constants are a special case, and they need to be treated differently. Depending on the numerical attribute domain, in case of a continuous real number domain for example, setting a numerical constant as an individual value will very likely have a very low support and also that would result and extremely large number of possible constants. Therefore, we split the attribute's domain into \emph{k} buckets and then check if any of the buckets present any gain in comparison with its correspondent numerical constant-free hypothesis.

For example if we test the hypothesis and we find support=100 and confidence=0.4:

\begin{center}
 \begin{math}isMarriedTo(x,y) \leftarrow hasAge(x,z)\end{math} 
\end{center}

and then we split z into three buckets:

\begin{itemize}
 \item \begin{math} k=1: z\in[0,20]\end{math}
 \item \begin{math} k=2: z\in(20,40]\end{math}
 \item \begin{math} k=3: z\in(40,\infty]\end{math}
\end{itemize}

we then test the hypothesis for each of the three buckets and obtain

\begin{itemize}

 \item \begin{math}isMarriedTo(x,y) \leftarrow hasAge(x,z), z\in[0,20]\end{math}	
    \newline support=40, confidence=0.1
 \item \begin{math}isMarriedTo(x,y) \leftarrow hasAge(x,z), z\in(20,40]\end{math}	
    \newline support=40, confidence=0.5
 \item \begin{math}isMarriedTo(x,y) \leftarrow hasAge(x,z), z\in(40,\infty]\end{math}
    \newline support=20, confidence=0.8

\end{itemize}

for k=2 and k=3, the hypothesis has significant gain by specifying numerical constants. Adding a relation to the body might produce totally different support and confidence distributions along the buckets. For example, if we add the relation hasChild(x,a), we could obtain other interesting rules:
\begin{itemize}
 \item \begin{math}isMarriedTo(x,y) \leftarrow hasAge(x,z)hasChild(x,a)\end{math}	
    \newline support=50, confidence=0.625
 \item \begin{math}isMarriedTo(x,y) \leftarrow hasAge(x,z)hasChild(x,a), z\in[0,20]\end{math}	
    \newline support=2, confidence=0.5
 \item \begin{math}isMarriedTo(x,y) \leftarrow hasAge(x,z)hasChild(x,a), z\in(20,40]\end{math}	
    \newline support=30, confidence=0.7
 \item \begin{math}isMarriedTo(x,y) \leftarrow hasAge(x,z)hasChild(x,a), z\in(40,\infty]\end{math}	
    \newline support=18, confidence=0.9
\end{itemize}

adding some relations might not bring any gain or even loss in confidence, but when bucketing per age, present a different distribution, 

Nevertheless, adding some relations with no correlation to the rule might not generate any interesting rules if  

\section{Contributions}
In this Thesis, we propose a pre-processing step to build a graph we call \graphname for each numerical property. In each graph, that has a numerical property as root, we first query the frequency distribution on the numerical attribute, then split them in \emph{k} buckets. Then we pick a set of \emph{c} categorical properties that can be joined with the root, and analyze how the distribution of sub-population created by joining them with the root is affected. Afterwards we try to combine each of the categories and see if they still produce interesting sub-populations, like in frequent set mining.

We also evaluate different heuristics and interestingness measures.

In a hypothesis containing a numerical attribute in the body, we can obtain a support and confidence value for each of the buckets, and 

With that, during the ILP algorithm, once we add one of the root properties, we can then search for the most interesting categorical properties that could result in different accuracy distributions. For every categorical property we can also suggest the most interesting constants and other categorical properties to be combined in a subcategory of both.

\section{Outline}

\begin{comment}
 The remainder of this thesis is structured as follows. In
Chapter~\ref{ch:technical_background}, we provide technical background on
MapReduce and BigTable. In Chapter~\ref{ch:related_work}, we present a
summary of previous work in the areas of duplicate and near-duplicate detection,
information retrieval on web archives, and MapReduce applications in graph
processing. Following that, we state our problem and describe solutions in
Chapter~\ref{ch:redundancy_control}. In Chapter~\ref{ch:mapreduce_impl}, we
describe an implementation of our solution using the MapReduce framework. In
Chapter~\ref{ch:experiments}, we present our experimental results. We conclude
this thesis and outline directions of future research in Chapter~\ref{ch:future_work}.
\end{comment}
